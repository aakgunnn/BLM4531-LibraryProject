\documentclass[12pt,a4paper]{article}
\usepackage[utf8]{inputenc}
\usepackage{geometry}
\usepackage{graphicx}
\usepackage{hyperref}
\usepackage{listings}
\usepackage{xcolor}
\usepackage{booktabs}
\usepackage{float}
\usepackage{caption}
\usepackage{longtable}
\usepackage{fancyhdr}
\usepackage{tocloft}

\geometry{margin=2.5cm}
\hypersetup{colorlinks=true, linkcolor=blue, urlcolor=blue}

% Kod stili
\lstset{
    basicstyle=\ttfamily\small,
    breaklines=true,
    frame=single,
    backgroundcolor=\color{gray!10},
    keywordstyle=\color{blue},
    commentstyle=\color{green!60!black},
    stringstyle=\color{orange},
    showstringspaces=false
}

\pagestyle{fancy}
\fancyhf{}
\rhead{Kutuphane Yonetim Sistemi}
\lhead{BLM4531 Final Projesi}
\rfoot{Sayfa \thepage}

\begin{document}

% ==================== KAPAK SAYFASI ====================
\begin{titlepage}
    \centering
    \vspace*{1cm}
    
    {\Large\textbf{ANKARA UNIVERSITESI}}\\[0.3cm]
    {\large Muhendislik Fakultesi}\\[0.3cm]
    {\large Bilgisayar Muhendisligi Bolumu}\\[2cm]
    
    {\Huge\textbf{KUTUPHANE YONETIM SISTEMI}}\\[0.5cm]
    {\Large Full-Stack Web Uygulamasi}\\[2cm]
    
    {\large\textbf{BLM4531 - Yazilim Muhendisligi Final Projesi}}\\[2cm]
    
    \begin{tabular}{ll}
        \textbf{Ogrenci:} & [ADI SOYADI] \\
        \textbf{Ogrenci No:} & [NUMARA] \\
        \textbf{Danisman:} & [HOCANIN ADI] \\
    \end{tabular}\\[2cm]
    
    \vfill
    
    % VIDEO LINKI - ONEMLI!
    \fbox{\parbox{0.8\textwidth}{
        \centering
        \textbf{PROJE TANITIM VIDEOSU}\\[0.3cm]
        \url{https://BURAYA-VIDEO-LINKINI-KOYUN}\\[0.2cm]
        \small (Video izinli ve herkese acik olmalidir)
    }}
    
    \vspace{1cm}
    {\large Ocak 2026}
\end{titlepage}

% ==================== ICINDEKILER ====================
\tableofcontents
\newpage

% ==================== OZET ====================
\section{Ozet}
Bu proje, ASP.NET Core Web API kullanilarak gelistirilen full-stack bir kutuphane yonetim sistemidir. Sistem, kitap yonetimi, kullanici yonetimi, odunc alma/iade islemleri ve istatistik raporlama gibi temel kutuphane islevlerini icermektedir.

\textbf{Kullanilan Teknolojiler:}
\begin{itemize}
    \item \textbf{Backend:} ASP.NET Core 8.0 Web API, Entity Framework Core, PostgreSQL
    \item \textbf{Web Frontend:} HTML5, CSS3, Bootstrap 5, JavaScript, Chart.js
    \item \textbf{Kimlik Dogrulama:} JWT (JSON Web Token)
\end{itemize}

% ==================== GIRIS ====================
\section{Giris ve Amac}

\subsection{Projenin Amaci}
Bu projenin temel amaci, modern yazilim gelistirme pratiklerini kullanarak kapsamli bir kutuphane yonetim sistemi gelistirmektir. Proje su hedefleri karsilamaktadir:

\begin{enumerate}
    \item Kitap kataloglama ve arama
    \item Kullanici kayit ve kimlik dogrulama
    \item Odunc alma taleplerinin yonetimi
    \item Admin paneli ile tam yonetim
    \item Istatistik ve raporlama
\end{enumerate}

\subsection{Kapsam}
Sistem iki kullanici rolunu desteklemektedir:
\begin{itemize}
    \item \textbf{Uye (Member):} Kitap goruntuleme, odunc talep etme, kendi odunclerini takip etme
    \item \textbf{Admin:} Tum islemler + kitap/kategori yonetimi, talep onaylama/reddetme, istatistik goruntuleme
\end{itemize}

% ==================== MIMARI ====================
\section{Sistem Mimarisi}

\subsection{Katmanli Mimari}
Proje, Clean Architecture prensiplerine uygun olarak katmanli bir yapida tasarlanmistir:

\begin{figure}[H]
\centering
\begin{tabular}{|c|}
\hline
\textbf{Controllers (API Layer)} \\
AuthController, BooksController, CategoriesController, \\
LoansController, AdminLoansController, StatisticsController \\
\hline
\textbf{Services (Business Logic Layer)} \\
AuthService, BookService, CategoryService, \\
LoanService, StatisticsService, JwtService \\
\hline
\textbf{Repositories (Data Access Layer)} \\
GenericRepository, UnitOfWork Pattern \\
\hline
\textbf{Data (Database Layer)} \\
LibraryDbContext, Entity Configurations \\
\hline
\textbf{PostgreSQL Database} \\
\hline
\end{tabular}
\caption{Katmanli Mimari Yapisi}
\end{figure}

\subsection{Proje Dosya Yapisi}
\begin{lstlisting}[language=bash]
Library.Net2/
|-- Controllers/          # API Endpointleri
|-- Services/             # Is mantigi
|-- Repositories/         # Veri erisim
|-- Data/                 # DbContext, Seeder
|-- Models/
|   |-- Domain/           # Entityler
|   |-- DTOs/             # Data Transfer Objects
|   |-- Enums/            # Enum tipleri
|-- wwwroot/              # Frontend dosyalari
|   |-- pages/            # HTML sayfalari
|   |-- js/               # JavaScript
|   |-- css/              # Stiller
|   |-- images/           # Gorseller
|-- Program.cs            # Uygulama giris noktasi
\end{lstlisting}

% ==================== VERITABANI ====================
\section{Veritabani Tasarimi}

\subsection{Entity-Relationship Diyagrami}
Sistem 4 ana entity icermektedir:

\begin{table}[H]
\centering
\begin{tabular}{|l|l|l|}
\hline
\textbf{Tablo} & \textbf{Aciklama} & \textbf{Iliskiler} \\
\hline
Users & Kullanicilar & 1:N Loans \\
Categories & Kitap kategorileri & 1:N Books \\
Books & Kitaplar & N:1 Category, 1:N Loans \\
Loans & Odunc kayitlari & N:1 User, N:1 Book \\
\hline
\end{tabular}
\caption{Veritabani Tablolari}
\end{table}

\subsection{Tablo Yapilari}

\subsubsection{Users Tablosu}
\begin{lstlisting}[language=SQL]
CREATE TABLE "Users" (
    "Id" SERIAL PRIMARY KEY,
    "FullName" VARCHAR(100) NOT NULL,
    "Email" VARCHAR(100) UNIQUE NOT NULL,
    "PasswordHash" TEXT NOT NULL,
    "Role" INTEGER NOT NULL  -- 0: Member, 1: Admin
);
\end{lstlisting}

\subsubsection{Categories Tablosu}
\begin{lstlisting}[language=SQL]
CREATE TABLE "Categories" (
    "Id" SERIAL PRIMARY KEY,
    "Name" VARCHAR(100) NOT NULL,
    "IsActive" BOOLEAN DEFAULT TRUE
);
\end{lstlisting}

\subsubsection{Books Tablosu}
\begin{lstlisting}[language=SQL]
CREATE TABLE "Books" (
    "Id" SERIAL PRIMARY KEY,
    "Title" VARCHAR(200) NOT NULL,
    "Author" VARCHAR(100) NOT NULL,
    "CategoryId" INTEGER REFERENCES "Categories"("Id"),
    "ISBN" VARCHAR(13),
    "PublishYear" INTEGER,
    "ImageUrl" TEXT,
    "IsAvailable" BOOLEAN DEFAULT TRUE
);
\end{lstlisting}

\subsubsection{Loans Tablosu}
\begin{lstlisting}[language=SQL]
CREATE TABLE "Loans" (
    "Id" SERIAL PRIMARY KEY,
    "UserId" INTEGER REFERENCES "Users"("Id"),
    "BookId" INTEGER REFERENCES "Books"("Id"),
    "LoanDate" TIMESTAMP NOT NULL,
    "DueDate" TIMESTAMP,
    "ReturnDate" TIMESTAMP,
    "Status" INTEGER NOT NULL,  -- Pending, Borrowed, Returned, Late
    "AdminNote" TEXT
);
\end{lstlisting}

% ==================== API DOKUMANTASYONU ====================
\section{API Endpoint Dokumantasyonu}

\subsection{Kimlik Dogrulama (Auth)}
\begin{longtable}{|l|l|p{6cm}|}
\hline
\textbf{Metod} & \textbf{Endpoint} & \textbf{Aciklama} \\
\hline
\endhead
POST & /api/Auth/register & Yeni kullanici kaydi \\
POST & /api/Auth/login & Giris, JWT token doner \\
GET & /api/Auth/me & Mevcut kullanici bilgisi \\
\hline
\caption{Auth API Endpointleri}
\end{longtable}

\subsection{Kitaplar (Books)}
\begin{longtable}{|l|l|p{6cm}|}
\hline
\textbf{Metod} & \textbf{Endpoint} & \textbf{Aciklama} \\
\hline
\endhead
GET & /api/Books & Tum kitaplari listele \\
GET & /api/Books/\{id\} & Kitap detayi \\
GET & /api/Books/search?query=x & Kitap ara \\
POST & /api/Books & Yeni kitap ekle (Admin) \\
PUT & /api/Books/\{id\} & Kitap guncelle (Admin) \\
DELETE & /api/Books/\{id\} & Kitap sil (Admin) \\
\hline
\caption{Books API Endpointleri}
\end{longtable}

\subsection{Kategoriler (Categories)}
\begin{longtable}{|l|l|p{6cm}|}
\hline
\textbf{Metod} & \textbf{Endpoint} & \textbf{Aciklama} \\
\hline
\endhead
GET & /api/Categories & Tum kategoriler \\
GET & /api/Categories/active & Aktif kategoriler \\
POST & /api/Categories & Kategori ekle (Admin) \\
PUT & /api/Categories/\{id\} & Guncelle (Admin) \\
DELETE & /api/Categories/\{id\} & Sil (Admin) \\
\hline
\caption{Categories API Endpointleri}
\end{longtable}

\subsection{Odunc Islemleri (Loans)}
\begin{longtable}{|l|l|p{6cm}|}
\hline
\textbf{Metod} & \textbf{Endpoint} & \textbf{Aciklama} \\
\hline
\endhead
GET & /api/Loans/my & Kullanicinin oduncleri \\
POST & /api/Loans & Odunc talebi olustur \\
POST & /api/Loans/\{id\}/return & Iade et \\
\hline
\caption{Loans API Endpointleri}
\end{longtable}

\subsection{Admin Odunc Yonetimi}
\begin{longtable}{|l|l|p{6cm}|}
\hline
\textbf{Metod} & \textbf{Endpoint} & \textbf{Aciklama} \\
\hline
\endhead
GET & /api/Admin/Loans & Tum oduncler \\
GET & /api/Admin/Loans/pending & Bekleyen talepler \\
GET & /api/Admin/Loans/late & Gecikmis oduncler \\
POST & /api/Admin/Loans/\{id\}/approve & Talebi onayla \\
POST & /api/Admin/Loans/\{id\}/reject & Talebi reddet \\
\hline
\caption{Admin Loans API Endpointleri}
\end{longtable}

\subsection{Istatistikler (Statistics)}
\begin{longtable}{|l|l|p{6cm}|}
\hline
\textbf{Metod} & \textbf{Endpoint} & \textbf{Aciklama} \\
\hline
\endhead
GET & /api/Statistics & Genel istatistikler \\
GET & /api/Statistics/categories & Kategori bazli \\
GET & /api/Statistics/monthly-loans & Aylik trend \\
GET & /api/Statistics/top-books & Populer kitaplar \\
\hline
\caption{Statistics API Endpointleri}
\end{longtable}

% ==================== FRONTEND ====================
\section{Frontend Gelistirme}

\subsection{Web Uygulamasi}
Web frontend, modern ve responsive bir tasarima sahiptir:

\begin{itemize}
    \item \textbf{Ana Sayfa:} Kitap listesi, arama, kategori filtreleme
    \item \textbf{Kitap Detay:} Kitap bilgileri, odunc alma butonu
    \item \textbf{Kullanici Paneli:} Odunclerim, profil bilgileri
    \item \textbf{Admin Paneli:} Dashboard, kitap/kategori/odunc yonetimi, istatistikler
\end{itemize}

\subsection{Kullanilan Teknolojiler}
\begin{itemize}
    \item Bootstrap 5 - Responsive tasarim
    \item Chart.js - Istatistik grafikleri
    \item Vanilla JavaScript - API iletisimi
\end{itemize}

\subsection{Sayfa Yapisi}
\begin{lstlisting}[language=bash]
wwwroot/
|-- pages/
|   |-- index.html           # Ana sayfa
|   |-- login.html           # Giris
|   |-- register.html        # Kayit
|   |-- book-detail.html     # Kitap detay
|   |-- my-loans.html        # Odunclerim
|   |-- admin-dashboard.html # Admin paneli
|   |-- statistics.html      # Istatistikler
|-- js/
|   |-- api.js               # API client
|   |-- admin.js             # Admin islemleri
|   |-- auth.js              # Kimlik dogrulama
|-- css/
|   |-- style.css            # Genel stiller
\end{lstlisting}


% ==================== EKRAN GORUNTULERI ====================
\section{Ekran Goruntuleri}

% Buraya ekran goruntuleri eklenecek
% \includegraphics[width=\textwidth]{screenshots/admin-dashboard.png}

\textit{Not: Ekran goruntuleri icin images/ klasorune gorseller eklenip bu bolum guncellenmelidir.}

\subsection{Web Uygulamasi Sayfalari}
\begin{itemize}
    \item Ana sayfa - Kitap listesi
    \item Admin Dashboard
    \item Istatistikler sayfasi
    \item Odunc yonetimi
    \item Giris ve Kayit sayfalari
\end{itemize}

% ==================== SONUC ====================
\section{Sonuc ve Degerlendirme}

\subsection{Tamamlanan Ozellikler}
\begin{itemize}
    \item JWT tabanli kimlik dogrulama sistemi
    \item Kitap ve kategori CRUD islemleri
    \item Odunc talep, onaylama, reddetme, iade sistemi
    \item Gecikmis odunc takibi
    \item Kapsamli istatistik ve raporlama (grafikler dahil)
    \item Responsive web arayuzu
\end{itemize}

\subsection{Ogrenilen Konular}
\begin{itemize}
    \item ASP.NET Core Web API gelistirme
    \item Entity Framework Core ile veritabani yonetimi
    \item JWT token tabanli guvenlik
    \item Clean Architecture ve katmanli mimari
    \item RESTful API tasarimi
    \item Frontend-Backend entegrasyonu
\end{itemize}

\subsection{Gelecek Gelistirmeler}
\begin{itemize}
    \item E-posta bildirimleri
    \item Kitap rezervasyon sistemi
    \item QR kod ile kitap tarama
    \item Coklu dil destegi
\end{itemize}

% ==================== KAYNAKLAR ====================
\section{Kaynaklar}
\begin{enumerate}
    \item Microsoft ASP.NET Core Documentation - \url{https://docs.microsoft.com/aspnet/core}
    \item Entity Framework Core - \url{https://docs.microsoft.com/ef/core}
    \item Chart.js - \url{https://www.chartjs.org}
    \item Bootstrap 5 - \url{https://getbootstrap.com}
    \item JWT.io - \url{https://jwt.io}
\end{enumerate}

\end{document}
